\markdownRendererDocumentBegin
\markdownRendererHeadingOne{Домашнє завдання №10}\markdownRendererInterblockSeparator
{}\markdownRendererUlBeginTight
\markdownRendererUlItem Записи \markdownRendererCodeSpan{Record} у \markdownRendererCodeSpan{AddressBook} зберігаються як значення у словнику. В якості ключів використовується значення \markdownRendererCodeSpan{Record.name.value}.\markdownRendererUlItemEnd 
\markdownRendererUlItem \markdownRendererCodeSpan{Record} зберігає об'єкт <Name> в окремому атрибуті.\markdownRendererUlItemEnd 
\markdownRendererUlItem \markdownRendererCodeSpan{Record} зберігає список об'єктів \markdownRendererCodeSpan{Phone} в окремому атрибуті.\markdownRendererUlItemEnd 
\markdownRendererUlItem \markdownRendererCodeSpan{Record} реалізує методи додавання/видалення/редагування об'єктів \markdownRendererCodeSpan{Phone}.\markdownRendererUlItemEnd 
\markdownRendererUlItem \markdownRendererCodeSpan{AddressBook} реалізує метод \markdownRendererCodeSpan{add\markdownRendererUnderscore{}record}, який додає <Record> у \markdownRendererCodeSpan{self.data}.\markdownRendererUlItemEnd 
\markdownRendererUlEndTight \markdownRendererInterblockSeparator
{}\markdownRendererHeadingTwo{Команди}\markdownRendererInterblockSeparator
{}\markdownRendererUlBeginTight
\markdownRendererUlItem \markdownRendererCodeSpan{hello} --- чат вітається.\markdownRendererUlItemEnd 
\markdownRendererUlItem \markdownRendererCodeSpan{set birthday} -- встановлює дату народження контакуту у форматі \markdownRendererCodeSpan{DD.MM.YYY}, наприклад \markdownRendererCodeSpan{set birthday Sergiy 12.12.1978}.\markdownRendererUlItemEnd 
\markdownRendererUlItem \markdownRendererCodeSpan{birthday of} -- Виводить на екран дату вказаного контакту, наприклад \markdownRendererCodeSpan{birthday of Sergiy}.\markdownRendererUlItemEnd 
\markdownRendererUlItem \markdownRendererCodeSpan{add} --- чат додає ім'я і телефон, приклад \markdownRendererCodeSpan{add Sergiy 0936564532}.\markdownRendererUlItemEnd 
\markdownRendererUlItem \markdownRendererCodeSpan{chage} --- чат змінює номер для відповідного контакту, приклад \markdownRendererCodeSpan{change Sergiy 0936564532 0634564545}.\markdownRendererUlItemEnd 
\markdownRendererUlItem \markdownRendererCodeSpan{phones} --- чат виводить номери телефонів контакту, приклад \markdownRendererCodeSpan{phone Sergiy}.\markdownRendererUlItemEnd 
\markdownRendererUlItem \markdownRendererCodeSpan{show all}--- чат показує усі контакти та їх номери, приклад \markdownRendererCodeSpan{show all}\markdownRendererUlItemEnd 
\markdownRendererUlItem \markdownRendererCodeSpan{remove} --- чат видаляє запис з вказаним іменем, приклад \markdownRendererCodeSpan{remove Sergiy}.\markdownRendererUlItemEnd 
\markdownRendererUlItem \markdownRendererCodeSpan{good bye}, \markdownRendererCodeSpan{good}, \markdownRendererCodeSpan{exit} --- чат прощається і завершує роботу і зберігає контакти у файл \markdownRendererCodeSpan{contacts.json}.\markdownRendererUlItemEnd 
\markdownRendererUlItem \markdownRendererCodeSpan{.} --- чат перериває свою роботу без попереджень і зберігає контакти у файл \markdownRendererCodeSpan{contacts.json}.\markdownRendererUlItemEnd 
\markdownRendererUlItem \markdownRendererCodeSpan{save} --- зберігає контакти у файл \markdownRendererCodeSpan{.json}, наприклад \markdownRendererCodeSpan{save contacts}.\markdownRendererUlItemEnd 
\markdownRendererUlItem \markdownRendererCodeSpan{load} --- завантажує книгу з контактами з файлу \markdownRendererCodeSpan{.json} в чат, наприклад \markdownRendererCodeSpan{load contacts}.\markdownRendererUlItemEnd 
\markdownRendererUlItem \markdownRendererCodeSpan{search} -- здійснює пошук за ключовою фразою, частиною номеру телефона чи дні народження, наприклад \markdownRendererCodeSpan{search 123}, або \markdownRendererCodeSpan{search Beth}.\markdownRendererUlItemEnd 
\markdownRendererUlEndTight \markdownRendererInterblockSeparator
{}\markdownRendererCodeSpan{
COMMANDS = \markdownRendererLeftBrace{}
    "hello": hello,
    "set birthday": set\markdownRendererUnderscore{}birthday,
    "birthday of": birthday,
    "add": add,
    "change": change,
    "phones of": phones,
    "show all": show\markdownRendererUnderscore{}all,
    "remove": remove,
    "good bye": good\markdownRendererUnderscore{}bye,
    "close": good\markdownRendererUnderscore{}bye,
    "exit": good\markdownRendererUnderscore{}bye,
    "save": save,
    "load": load,
    "search": search,
\markdownRendererRightBrace{}
}\markdownRendererDocumentEnd